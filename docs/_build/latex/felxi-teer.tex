%% Generated by Sphinx.
\def\sphinxdocclass{report}
\documentclass[letterpaper,10pt,english]{sphinxmanual}
\ifdefined\pdfpxdimen
   \let\sphinxpxdimen\pdfpxdimen\else\newdimen\sphinxpxdimen
\fi \sphinxpxdimen=.75bp\relax
\ifdefined\pdfimageresolution
    \pdfimageresolution= \numexpr \dimexpr1in\relax/\sphinxpxdimen\relax
\fi
%% let collapsible pdf bookmarks panel have high depth per default
\PassOptionsToPackage{bookmarksdepth=5}{hyperref}

\PassOptionsToPackage{warn}{textcomp}
\usepackage[utf8]{inputenc}
\ifdefined\DeclareUnicodeCharacter
% support both utf8 and utf8x syntaxes
  \ifdefined\DeclareUnicodeCharacterAsOptional
    \def\sphinxDUC#1{\DeclareUnicodeCharacter{"#1}}
  \else
    \let\sphinxDUC\DeclareUnicodeCharacter
  \fi
  \sphinxDUC{00A0}{\nobreakspace}
  \sphinxDUC{2500}{\sphinxunichar{2500}}
  \sphinxDUC{2502}{\sphinxunichar{2502}}
  \sphinxDUC{2514}{\sphinxunichar{2514}}
  \sphinxDUC{251C}{\sphinxunichar{251C}}
  \sphinxDUC{2572}{\textbackslash}
\fi
\usepackage{cmap}
\usepackage[T1]{fontenc}
\usepackage{amsmath,amssymb,amstext}
\usepackage{babel}



\usepackage{tgtermes}
\usepackage{tgheros}
\renewcommand{\ttdefault}{txtt}



\usepackage[Bjarne]{fncychap}
\usepackage{sphinx}

\fvset{fontsize=auto}
\usepackage{geometry}


% Include hyperref last.
\usepackage{hyperref}
% Fix anchor placement for figures with captions.
\usepackage{hypcap}% it must be loaded after hyperref.
% Set up styles of URL: it should be placed after hyperref.
\urlstyle{same}

\addto\captionsenglish{\renewcommand{\contentsname}{Contents:}}

\usepackage{sphinxmessages}
\setcounter{tocdepth}{1}



\title{Felxi\sphinxhyphen{}TEER}
\date{Feb 17, 2022}
\release{1.0}
\author{MD Anderson}
\newcommand{\sphinxlogo}{\vbox{}}
\renewcommand{\releasename}{Release}
\makeindex
\begin{document}

\pagestyle{empty}
\sphinxmaketitle
\pagestyle{plain}
\sphinxtableofcontents
\pagestyle{normal}
\phantomsection\label{\detokenize{index::doc}}



\chapter{TEER Theory}
\label{\detokenize{pages/TEER-Theory:teer-theory}}\label{\detokenize{pages/TEER-Theory::doc}}
\sphinxAtStartPar
!{[}{]}(\sphinxurl{https://github.com/mdanderson03/Felxi-TEER/blob/11124297eae28fc72f78af3b5448df1af65e0caa/passive\%20diffusion.png})

\sphinxAtStartPar
The typical use of TEER is with transwells. In these, monolayers of cells are formed and ions can passively diffuse through the gaps between cells.

\sphinxAtStartPar
!{[}{]}(\sphinxurl{https://github.com/mdanderson03/Felxi-TEER/blob/11124297eae28fc72f78af3b5448df1af65e0caa/passive\%20diffusion2.png})

\sphinxAtStartPar
Lets start forming a model on this. So if a gap has a tight junction formed, it blocks the ion from going through it. If it lacks a tight junction, then the ion can go on through. Lets translate that into a simple electrical circuit model.

\sphinxAtStartPar
!{[}{]}(\sphinxurl{https://github.com/mdanderson03/Felxi-TEER/blob/11124297eae28fc72f78af3b5448df1af65e0caa/passive\%20diffusion3.png})

\sphinxAtStartPar
So we replace tight junctions with open circuits and given gaps restrict the rate that charged particles can pass through the monolayer, lets say they are  resistors. So \sphinxstylestrong{tight junctions = open circuit} and \sphinxstylestrong{gaps = resistors}

\sphinxAtStartPar
\sphinxstylestrong{Ohm’s Law}

\sphinxAtStartPar
\$\$V = I*R\$\$ (1)
\sphinxstylestrong{Parallel Resistance Equivalency}

\sphinxAtStartPar
\$\$R\_\{eq\} = dfrac\{1\}\{(sum\{dfrac\{1\}\{R\_g\}\})\}\$\$(2)
\$\$R\_\{1\} approx R\_\{2\} approx R\_\{3\} approx … approx R\_\{G\}\$\$(3)
Sub {[}3{]} into {[}2{]}
\$\$R\_\{eq\} = dfrac\{1\}\{(dfrac\{1\}\{R\_G\}*N\_\{G\})\}=dfrac\{R\_\{G\}\}\{N\_\{G\}\}\$\$(4)
\$\$N\_\{CJ\}=N\_\{TJ\}+N\_\{G\} to N\_\{G\} = N\_\{CJ\}\sphinxhyphen{}N\_\{TJ\}\$\$(5)
Sub (3) into (4) and taylor expand
\$\$R\_\{eq\}=dfrac\{R\_\{G\}\}\{N\_\{CJ\}*(1\sphinxhyphen{}dfrac\{ N\_\{TJ\}\} \{N\_\{CJ\}\})\} to dfrac\{R\_\{eq\}*N\_\{CJ\}\}\{R\_\{G\}\}
= dfrac\{1\}\{(1\sphinxhyphen{}dfrac\{N\_\{TJ\}\}\{N\_\{CJ\}\})\} approx 1+ dfrac\{N\_\{TJ\}\}\{N\_\{CJ\}\}\$\$(6)
Thus
\$\$R\_\{eq\} propto N\_\{TJ\}\$\$(7)

\sphinxAtStartPar
r”“”
.. math:: w\_k\textasciicircum{}* = min\_\{w\_k\} ell\_k(w\_k) + lambdaleft(alpha||w\_k||\_1
\begin{itemize}
\item {} 
\sphinxAtStartPar
frac\{1\}\{2\}(1\sphinxhyphen{}alpha) ||w\_k||\textasciicircum{}2right)

\end{itemize}

\sphinxAtStartPar
“”“


\chapter{Indices and tables}
\label{\detokenize{index:indices-and-tables}}\begin{itemize}
\item {} 
\sphinxAtStartPar
\DUrole{xref,std,std-ref}{genindex}

\item {} 
\sphinxAtStartPar
\DUrole{xref,std,std-ref}{modindex}

\item {} 
\sphinxAtStartPar
\DUrole{xref,std,std-ref}{search}

\end{itemize}



\renewcommand{\indexname}{Index}
\printindex
\end{document}